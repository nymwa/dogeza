\documentclass[14pt]{beamer}

% font
\usepackage{fontspec}
\setmainfont[Ligatures=TeX]{IPAPGothic}
\usepackage{xeCJK}
\setCJKmainfont{IPAPGothic}
\newfontfamily{\liberation}{Liberation Sans}
\newfontfamily{\notosans}{Noto Sans CJK JP}

% graphic
\usepackage{graphics}
\usepackage{graphicx}
\usepackage{color}
\usepackage{xcolor}
\usepackage{colortbl}
\usepackage{tcolorbox}
\definecolor{mygray}{rgb}{0.1, 0.1, 0.1}

% tikz
\usepackage{tikz}
\usetikzlibrary{automata}
\usetikzlibrary{arrows}
\usetikzlibrary{arrows.meta}
\usetikzlibrary{positioning}
\usetikzlibrary{intersections, calc}
\usetikzlibrary{decorations}
\usetikzlibrary{decorations.markings}
\usetikzlibrary{decorations.pathreplacing,angles,quotes}
\usetikzlibrary{fit}
\usetikzlibrary{math}
\usetikzlibrary{shapes}
\usepackage{pgfplots}
\usepackage{bchart}

% href
\usepackage{hyperref}
\hypersetup{
	colorlinks=true,
	linkcolor=cyan,
	filecolor=cyan,
	urlcolor=cyan,
	pdfnewwindow=true}

% beamer
\usepackage{bxdpx-beamer}
\usetheme{Boadilla}
\setbeamertemplate{navigation symbols}{}
\setbeamercovered{transparent}
\setbeamertemplate{frametitle}{%
	\vspace{0.1em}
	\usebeamerfont{frametitle}\insertframetitle%
	\par
	\rule[0.5\baselineskip]{0.9\paperwidth}{0.4pt}%
	\vspace{-0.5em}}
\setbeamertemplate{footline}{
	\hfill
	\usebeamercolor[fg]{page number in head/foot}
	\usebeamerfont{page number in head/foot}
	{\small \insertframenumber}
	\kern1em\vskip5pt
}
\setbeamercolor{footline}{fg=black,bg=black}

% itemize
\usepackage{enumitem}
\setitemize{itemsep=0.3em}
\setlength\leftmargini{20pt}
\setlength\leftmarginii{20pt}
\setlength\leftmarginiii{20pt}
\setlength\leftmarginiv{20pt}
\setlist[itemize,1]{label=$\color{blue}\bullet$}
\setlist[itemize,2]{label=$\color{orange}\triangleright$}
\setlist[itemize,3]{label=$\color{gray}\bullet$}
\setlist[itemize,4]{label=$\color{red}\triangleright$}
\setlist[itemize,5]{label=$\color{gray}\bullet$}
\setlist[itemize,6]{label=$\color{red}\triangleright$}
\setlist[itemize,7]{label=$\color{yellow}\bullet$}
\setlist[itemize,8]{label=$\color{pink}\triangleright$}
\setlist[itemize,9]{label=$\color{black}\bullet$}

% math
\usepackage{amsmath,amssymb,amsthm}
\usepackage{bm}
\setbeamertemplate{theorems}[numbered]
\theoremstyle{definition}
\newtheorem{thm}{定理}
\newtheorem{lem}{補題}
\newtheorem{prop}{命題}
\newtheorem{dfn}{定義}
\setbeamertemplate{theorem begin}{{
	\inserttheoremheadfont
	\inserttheoremname
	\inserttheoremnumber
	\inserttheorempunctuation
}}
\setbeamertemplate{theorem end}{}

% other
\usepackage{caption}
\usepackage{cancel}
\usepackage{epigraph}
\usepackage{fancybox}
\usepackage{here}
\usepackage{makecell}
\usepackage{setspace}
\usepackage{scrextend}
\usepackage{svg}
\usepackage{ulem}
\usepackage{multirow}

\begin{document}

\begin{frame}
	\begin{center}
		{\LARGE \color{cyan} nymwa\kern.0emさんはなぜ\\土下座大田区一周を\\しなければならないのか} \\
		\vspace{1em}
		{\scriptsize 〜トキポナが正則言語ではないことの証明〜}
	\end{center}
\end{frame}


\begin{frame}
	\frametitle{すべてはこのツイートから始まった}

	\begin{figure}[h]
		\centering
		\includegraphics[width=10cm]{tweet1.png}
	\end{figure}

\end{frame}


\begin{frame}
	\frametitle{自らの正しさを信じ切る\kern.0emnymwa}

	\begin{figure}[h]
		\centering
		\includegraphics[width=10cm]{tweet2.png}
	\end{figure}

\end{frame}


\begin{frame}
	\frametitle{妄言の末路}

	\begin{figure}[h]
		\centering
		\includegraphics[width=10cm]{tweet3.png}
	\end{figure}

\end{frame}


\begin{frame}
	\frametitle{示したいこと}

	\begin{tcolorbox}[
			colframe=orange!80,
			colback=orange!20,
			sharp corners]
		\begin{prop}\label{prop:pona}
			トキポナは正則言語である.
		\end{prop}

		\begin{thm}
			命題\ref{prop:pona}は偽である.
		\end{thm}
	\end{tcolorbox}

\end{frame}


\begin{frame}
	\frametitle{概要}

	\begin{itemize}
		\item 次のような流れです
			\begin{itemize}
				\item 正則言語・文脈自由言語・文脈依存言語とは何か?
				\item トキポナは文脈依存言語である証明
				\item トキポナは正則言語ではない!
					\begin{itemize}
						\item 文脈依存なら正則ではない
					\end{itemize}
			\end{itemize}
		\item 最後に土下座大田区一周の概要を説明します
	\end{itemize}

\end{frame}


\begin{frame}
	\frametitle{形式言語}
\end{frame}


\begin{frame}
	\frametitle{チョムスキー階層}
\end{frame}


\begin{frame}
	\frametitle{正則言語}
\end{frame}


\begin{frame}
	\frametitle{文脈自由言語}
\end{frame}


\begin{frame}
	\frametitle{文脈依存言語}
\end{frame}


\begin{frame}
	\frametitle{なぜトキポナが正則と思ったのか?}
\end{frame}


\begin{frame}
	\frametitle{コピー言語}
\end{frame}


\begin{frame}
	\frametitle{コピー言語は文脈自由でない}
\end{frame}


\begin{frame}
	\frametitle{コピー言語は文脈依存}
\end{frame}


\begin{frame}
	\frametitle{反復疑問は文脈依存}
\end{frame}


\begin{frame}
	\frametitle{トキポナは文脈依存}
\end{frame}


\begin{frame}
	\frametitle{トキポナは弱文脈依存文法}
\end{frame}


\begin{frame}
	\frametitle{トキポナは正則でないため…}
\end{frame}


\begin{frame}
	\frametitle{土下座大田区一周の概要}

	\begin{itemize}
		\item 多摩川 - 羽田 区間
			\begin{itemize}
				\item ゴムボートで土下座川下り
			\end{itemize}
		\item 天空橋 - 天王洲アイル 区間
			\begin{itemize}
				\item 土下座モノレール
			\end{itemize}
		\item 品川 - 目黒 区間
			\begin{itemize}
				\item 土下座山手線
			\end{itemize}
		\item 目黒 - 多摩川 区間
			\begin{itemize}
				\item 土下座目黒線
			\end{itemize}
		\item 乗り換えで歩くのは許してほしい
	\end{itemize}

\end{frame}

\end{document}

